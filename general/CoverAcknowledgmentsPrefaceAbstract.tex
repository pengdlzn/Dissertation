%-------------------
% WUP class commands
%-------------------

\TitleHalf{An Optimization-Based Approach \\ 
	for Continuous Map Generalization \\ (manuscript, \today)
}
\Title{An Optimization-Based Approach \\ 
	for Continuous Map Generalization}
\SubTitle{}
\Author{Dongliang Peng}

\Faculty{Fakult\"at f\"ur Mathematik und Informatik}
%\Faculty{Faculty of Mathematics and Computer Science}
\YearSubmission{2018}
\YearPublication{2019}
\Advisor{Prof.\ Dr.\ Alexander Wolff and Prof.\ Dr.\ Dirk Burghardt}
\FreeTextImprint{
%\\
%\setlength{\tabcolsep}{0em}	
%\begin{tabular}{x{2.5cm}l}
%Supervisors:	    & ~Prof.\ Dr.\ Alexander Wolff\\
%					& ~Prof.\ Dr.\ Jan-Henrik Haunert
%					\vspace{4pt}\\ 
%Reviewers:			& ~Prof.\ Dr.\ Alexander Wolff\\
%					& ~Prof.\ Dr.\ Dirk Burghardt
%					\vspace{4pt}\\
%Defense Examiners:	& ~Prof.\ Dr.\ Andreas N\"uchter\\
%					& ~Prof.\ Dr.\ Jan-Henrik Haunert\\
%					& ~Prof.\ Dr.\ Alexander Wolff
%					\vspace{4pt}\\
%\multicolumn{2}{l}{Defended on December 21, 2017}					
%\end{tabular}
}



\CoverDesign{Max Mustermann}
\FreeTextDesign{a free text for design}

\ISBNprint{978-3-95826-104-4}
\ISBNonline{978-3-95826-105-1}
\URN{urn:nbn:de:bvb:20-opus-174427}

\Dedication{%
the dedication
}
\Acknowledgments{%
}


\Preface{%

In his thesis, Dongliang Peng investigates computational aspects of
a fundamental problem in cartography called \emph{generalization}.
Cartographic generalization is the problem that has to be solved
when producing a small-scale map (say, at scale $1:15{,}000$) from data
collected at a larger scale (say, $1:15{,}000$).  Traditionally, this
has been a labor-intensive task that required highly-skilled
cartographers with expertise in both geodesy and map-making.  How
should a row of ten houses be represented if the distances between
them are too small to be represented correctly at the desired scale?
By fewer, say, five houses?  Or rather by a block of houses?

Dongliang attacks this and several other special cases of
cartographic generalization in a modern setting.  Think of online
maps that users can (and want to!) zoom and pan with their finger
tips.  Other than traditional paper maps that were produced and
printed for a fixed, small number of scales, online maps should
ideally be available at \emph{any} scale (within a reasonable
interval).  The problem of computing and displaying maps at
arbitrary scales is called \emph{continuous generalization}.  Some
aspects of continuous generalization (such as the example with the
houses above) are inherently discrete, others (such as a wiggly
river) can be treated more easily in a continuous fashion.

In several chapters of this thesis, Dongliang approaches continuous
generalization problems in an innovative, yet formal way by using
powerful tools from mathematical optimization such as integer linear
programming.  The main advantage of such an approach is that it
helps to get the \emph{model} right.  Many problems can be cast into
integer linear programs, which allows us to solve at least small
examples to optimality---according to our model.  If these
solutions don't turn out as expected, we know that 
%our problem formulation, 
our model is wrong---not the algorithm.  In contrast,
when using only heuristics, solutions can come with errors of two
types; those caused by the heuristic and those caused by the model.
In this case, it is hard to understand and fix the errors.

In Chapter~2, Dongliang compares two potentially exact methods for
the well-known area-aggregation problem.  He asks for an optimal
sequence of operations that aggregates, step-by-step, many
``patches'' of a detailed map into a single region of a coarser map.
Dongliang uses a graph-based model in which he finds shortest paths
(aka optimal aggregation sequences) using the \Astar algorithm
and integer linear programming. The latter turns out to be much slower.
He also compares the \Astar algorithm to an obvious greedy
approach, which is surprisingly good given its simplicity.  He also
identifies a problem with his model.

Chapter~3 treats a related problem: how to best transit between
two levels of administrative boundaries for zooming?  The
input consists of the two drawings at start and target scale, and
the task is to mediate between the two in a continuous and
topologically safe way. 
%that is, without introducing
%self-intersections or other degeneracies.  
Dongliang finds the first
topologically safe method of solving this problem based on
\emph{compatible triangulations} (a tool from computational
geometry).  Unfortunately, the existing algorithm for compatible
triangulations sometimes introduces strong distortion locally.  A
way out may consist in choosing the Steiner points for the
compatible triangulations more carefully, but this is left as an
open problem.

In Chapter~4, Dongliang deals with the generalization of buildings.
He shows how to generalize building footprints continuously such
that well-defined blocks of buildings appear when the user of a
digital map zooms out far enough.  Here, only the building
footprints are given.  The approach animates a growing-process
between the start scale and a target scale that can be set by the
user.  The algorithm computes a drawing for the target scale by a
wisely chosen sequence of dilations and erosions
(which have been used for building simplification before).  
In order to animate the
growing-process in a continuous fashion, the buildings are expanded
in a simple way: by moving their boundaries at constant speed and by
clipping any part that leaves the target footprint.  In order to
guarantee a certain minimum distance between two buildings, the
algorithm builds bridges between close-enough buildings.  
The computation of the bridges is based on 
a minimum spanning tree of the buildings.  
The resulting animations look very natural.

In Chapter~5, Dongliang explores an important subproblem that often
appears in continuous generalization: how to ``morph'' a polygonal
chain from a start to a target scale.  He uses an existing algorithm
to define a correspondence between the two chains and then computes
trajectories for each pair of corresponding vertices.  
The aim is to find trajectories such that 
the angles and the edge lengths change 
in a uniform and continuous fashion---if possible.  
Dongliang applies least-squares adjustment in order to
gradually move the chain from its start configuration to the target
chain.  While the results on real-world data look quite good,
Dongliang also found artificial examples where self-intersection and
numerical problems occur.  Hence, least squares is probably not the
ideal method for this problem.

Finally, in Chapter~6, Dongliang presents a case study to highlight
the difficulties when working with geographic data naively.  As a
concrete example, he considers the problem of finding, in a set of
$n$ points in the plane, all pairs of points that are closer to each
other than a given threshold~$\varepsilon$.  He compares three
approaches for this problem, the text-book approach based on the
sweep-line paradigm, a Delaunay-triangulation-based approach, and a
simple grid-based approach.  While the grid-based approach wins in
terms of runtime, the different implementations of the sweep-line
algorithms are what makes the comparison interesting---a lot
depends on how the library methods are implemented, 
so the programmer should always read the fine print.

In his thesis, Dongliang exemplifies the optimization-based approach
at various continuous generalization problems and demonstrates the
strength of this approach.  I~see this as a very valuable
contribution to GIScience where very all too often, problems are not
modeled properly, and then algorithms are devised that happen to
produce good-looking results on a small set of benchmark instances.
I~hope that many readers of this thesis will get inspired by
Dongliang's way of tackling spatial problems!

\bigskip
\bigskip
%\lineskip


\noindent
Alexander Wolff\\
Chair of Algorithms, 
Complexity and Knowledge-Based Systems\\
Faculty of Mathematics and Computer Science\\
University of W\"urzburg
}


\Abstract{	
Maps are the main tool to represent geographical information. 
Geographical information is usually scale-dependent, 
so users need to have access to maps at different scales.
In our digital age, the access is realized by zooming.
As discrete changes during the zooming tend to distract users,
smooth changes are preferred.
This is why some digital maps 
are trying to make the zooming as continuous as they can.
The process of producing maps at different scales
with smooth changes 
is called \emph{continuous map generalization}.

In order to produce maps of high quality,
cartographers often take into account additional requirements.
These requirements are transferred to models 
in map generalization.
Optimization for map generalization is important 
not only because it finds optimal solutions 
in the sense of the models,
but also because it helps us 
to evaluate the quality of the models.
Optimization, however, becomes more delicate
when we deal with \emph{continuous} map generalization.
In this area, there are requirements 
not only for a specific map
but also for relations between maps at difference scales. 
This thesis is about continuous map generalization
based on optimization.

First, we show the background of our research topics.
Second, we find optimal sequences 
for aggregating land-cover areas.
We compare the \Astar algorithm and integer linear programming
in completing this task.
Third, we continuously generalize 
county boundaries to provincial boundaries
based on compatible triangulations.
We morph between the two sets of boundaries, 
using dynamic programming to 
compute the correspondence.
Fourth, we continuously generalize buildings to built-up areas 
by aggregating and growing.
In this work, we group buildings with the help of a
minimum spanning tree.
Fifth, we define vertex trajectories that allow us
to morph between polylines.
We require that both the angles and the edge lengths 
change linearly over time.
As it is impossible to fulfill all of these requirements
simultaneously, 
we mediate between them using least-squares adjustment.
Sixth, we discuss the performance 
of some commonly used data structures 
for a specific spatial problem.
Last, we conclude this thesis 
and present open problems.
}


\AbstractGerman{

\subsubsection{Optimierung 
für die kontinuierliche Generalisierung von Landkarten}

Landkarten sind das wichtigste Werkzeug zur Repräsentation geografischer
Information.  Unter Generalisierung von Landkarten versteht man die
Aufbereitung von geografischer Information aus detaillierten Daten zur
Generierung von kleinmaßstäbigen Karten.  Nutzer von Online-Karten
zoomen oft in eine Karte hinein oder aus einer Karte heraus, um mehr
Details bzw. mehr Überblick zu bekommen.  Kontinuierliche
Generalisierung von Landkarten versucht die Änderungen zwischen
verschiedenen Maßstäben stetig zu machen, was wichtig ist, um Nutzern
eine angenehme Zoom-Erfahrung zu bieten.

Um eine qualitativ hochwertige kontinuierliche Generalisierung zu
erreichen, kann man wichtige Aspekte bei der Generierung von
Online-Karten optimieren.  In diesem Buch haben wir Optimierung bei der
Generalisierung von Landnutzungskarten, von administrativen Grenzen,
Gebäuden und Küstenlinien eingesetzt.  Unsere Experimente zeigen, dass
die kontinuierliche Generalisierung von Landkarten in der Tat von
Optimierung profitiert.


%###English Version
%Maps are the main tool to represent geographical information.
%Map generalization is the process of extracting and arranging
%geographical information from detailed data 
%in order to produce maps at smaller scales.
%Users of online maps often zoom in and out 
%to get more details or more overview, respectively.
%Continuous map generalization tries to make the changes
%between different scales smooth, which is essential
%to provide users with comfortable zooming experience.
%
%In order to achieve continuous map generalization with high quality,
%we optimize some important aspects when producing online maps.
%In this book, we have used optimization in the generalization of
%land-cover areas, administrative boundaries, buildings, and coastlines.
%According to our experiments, continuous map generalization indeed
%benefits from optimization.
}



%%% Local Variables:
%%% mode: latex
%%% TeX-master: "../dissertation"
%%% End:
