%-------------------
% WUP class commands
%-------------------

\TitleHalf{An Optimization-Based Approach \\ 
	for Continuous Map Generalization 
}
\Title{An Optimization-Based Approach \\ 
	for Continuous Map Generalization}
\SubTitle{}
\Author{Dongliang Peng}


\Faculty{Faculty of Mathematics and Computer Science}
\YearSubmission{2017}
\YearPublication{2018}
\FreeTextImprint{
\\
\setlength{\tabcolsep}{0em}	
\begin{tabular}{x{2.5cm}l}
Supervisors:	    & ~Prof. Dr. Alexander Wolff\\
					& ~Prof. Dr. Jan-Henrik Haunert
					\vspace{4pt}\\ 
Reviewers:			& ~Prof. Dr. Alexander Wolff\\
					& ~Prof. Dr. Dirk Burghardt
					\vspace{4pt}\\
Defense Examiners:	& ~Prof. Dr. Andreas N\"uchter\\
					& ~Prof. Dr. Jan-Henrik Haunert\\
					& ~Prof. Dr. Alexander Wolff
					\vspace{4pt}\\
\multicolumn{2}{l}{Defended on December 21, 2017}					
\end{tabular}
}



\CoverDesign{Max Mustermann}
\FreeTextDesign{a free text for design}

\ISBNprint{978-3-95826-XXX-X}
\ISBNonline{978-3-95826-XXX-X}
\URN{urn:nbn:de:bvb:20-opus-XXXXXX}

\Acknowledgments{%
I would like to thank my supervisor, 
Prof. Dr. Alexander Wolff,
who we often call Sascha.
Sascha is very patient to me.
When I first came to Germany, I could barely speak English.
Sascha had to spend a lot of effort to understand me
when we were doing research work together.
He always encouraged me to ask questions.
The research work was challenging,
but I enjoyed doing my research here.
I am so lucky to have Sascha as my supervisor.
He even negotiated with my landlords.

I also thank my second supervisor, Prof. Dr. Jan-Henrik Haunert.  In
his lecture Algorithms for GIS, he taught my many fundamentals of
GIScience.  Later he moved to other universities and invited me the
visit him there.

I would like to thank Prof. Dr. Min Deng
for helping me get the chance of
pursuing my PhD in University of W\"urzburg.

AW: Don't forget Guillaume Touya!

I am grateful to Dr. Krzysztof Fleszar for 
helping me a lot during my PhD study.
Whenever he found out that I had problems to understand a paper,
he read that paper and then explained it to me.
%On the day of my defense, he sent his last set of comments 
%about my slides at about 3:00 am.
He helped me move twice.
On the day before my defense, he worked as hard as me
to read my slides and to give me feedback.
He has always invited me to join his parties, 
I am grateful to him for his invitations. 

I am very grateful to the University of W\"urzburg 
because I had the possibility to freely take many courses.
In order to gain sufficient credits for 
obtaining a PhD degree of Computer Science,
I took some courses
(e.g., \emph{Theoretical Computer Science}
and \emph{Algorithms for Data Structures})
from our own faculty, 
Faculty of Mathematics and Computer Science.
%
In order to improve my language skills,
I took many courses
offered by the Language Center
(e.g., \emph{English for the Natural Sciences} 
and \emph{General Language Exercise of German})
and by Faculty of Arts 
(e.g., \emph{Introduction to English Linguistics}).

It was nice that our group always had lunch together
so that we had plenty of chances to learn from each other.
Indeed, I sometimes took advantage of these lunches
to get suggestions regarding my research work.

thanks to my girlfriend.

%My sincerest thank is meant to my enthusiastic supervisor, 
%[Name]. My PhD has been an amazing experience and I thank 
%[Name] wholeheartedly, not only for his tremendous academic 
%support, but also for giving me so many wonderful 
%opportunities. I am particularly indebted to [Name] for his 
%constant faith in my lab work, all his contributions of 
%time, ideas, and funding to make my Ph.D. experience 
%productive and stimulating. 
%I am also hugely appreciative of [Name], especially for 
%sharing his (computing???) expertise so willingly, and for 
%being so dedicated to his role as my secondary supervisor, 
%and for his support when so generously hosting me in [City].
%A special acknowledgement goes to YOUR Collaborative team 
%[Name]. I praise the enormous amount of help and support by 
%[Name] throughout my stay in Paris. 
%Dozens of people have helped and taught me immensely at 
%Wuerzburg University. Special mention goes to cristoph??, 
%[Name], and [Name], for going far beyond the call of duty. 
%To [Name], for encouraging me to embark on the computer 
%science\Geography?? path, and for inspiring conversations. 
%And to [your other members], for nurturing my enthusiasm for 
%computer science. The group has been a source of friendships 
%as well as good advice and collaboration. I have very fond 
%memories of Squash time here. 
%For this dissertation I would like to thank my reading 
%committee members: [Name], [Name], and [Name] for their 
%time, interest, and helpful comments. 
%My time at Wuerzburg was made enjoyable and enriched in 
%large part due to the many friends and groups that became a 
%part of my life. I am grateful for time spent with flatmates 
%and friends, for my Football Team, Hotpot group, and Skiing 
%buddies. 
%Lastly, I would like to thank my family for all their love 
%and encouragement. For my parents who raised me with a love 
%of science and supported me in all my pursuits. And most of 
%all for my loving, supportive, and encouraging fiancée Fang 
%whose faithful support during the final stage of this Ph.D. 
%is so appreciated. Thank you.
}
\Preface{%
	
	\lipsum[1-8]
}
\Abstract{	
Maps are the main tool to represent geographical information. 
As geographical information is usually scale-dependent, 
users need to have access to maps at different scales.
In our digital age, people often zoom in or out 
to read maps at different levels.
As discrete changes tend to distract users,
smooth changes are preferred.
This is why some digital maps such as Google Maps 
are trying to make zooming as continuous as they can.
The process of producing maps at different scales
with smooth changes 
is called \emph{continuous map generalization}.
Ideally, there
should be no discrete change in 
continuous map generalization. 
However, the term is also used
when the discrete changes are small enough not to be noticed.

In order to produce maps with high quality,
cartographers often take into account requirements.
These requirements are transferred to models 
in map generalization.
Optimization for map generalization is important 
not only because it finds optimal solutions in the sense of the 
models,
but also because it evaluates the quality of the models.
Optimization becomes more delicate
when we deal with \emph{continuous} map generalization.
In this area, there are requirements 
not only for a specific map, 
but also for relations between maps. 
%
My PhD thesis is about continuous map generalization
based on optimization. 
First, we find optimal sequences for aggregating land-cover 
areas (see Figure 1).
We compare the \Astar algorithm and integer linear programming
in completing this task.
%
Second, we generalize county boundaries to
provincial boundaries.
We morph between the two sets of boundaries, 
where dynamic programming is used to 
compute corresponding points between boundaries.
%
Third, we generalize buildings to built-up areas 
by aggregating and growing (see Figure 2).
In this work, we group buildings based on
minimum spanning tree. 
%
Fourth, we define moving trajectories
for morphing between polylines.
We require that both the lengths of edges and the angles 
between edges should change linearly.
As it is impossible to fulfill all the requirements,
we make a compromise between them 
using least-squares adjustment.
%
Fifth, we discuss the performance 
of some data structures 
(e.g., SortedDictionary and SortedSet) 
for a specific spatial problem.
}