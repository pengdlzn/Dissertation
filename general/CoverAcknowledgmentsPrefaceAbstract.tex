%-------------------
% WUP class commands
%-------------------

\TitleHalf{An Optimization-Based Approach \\ 
	for Continuous Map Generalization 
}
\Title{An Optimization-Based Approach \\ 
	for Continuous Map Generalization}
\SubTitle{}
\Author{Dongliang Peng}


\Faculty{Faculty of Mathematics and Computer Science}
\YearSubmission{2017}
\YearPublication{2018}
\FreeTextImprint{
\\
\setlength{\tabcolsep}{0em}	
\begin{tabular}{x{2.5cm}l}
Supervisors:	    & ~Prof. Dr. Alexander Wolff\\
					& ~Prof. Dr. Jan-Henrik Haunert
					\vspace{4pt}\\ 
Reviewers:			& ~Prof. Dr. Alexander Wolff\\
					& ~Prof. Dr. Dirk Burghardt
					\vspace{4pt}\\
Defense Examiners:	& ~Prof. Dr. Andreas N\"uchter\\
					& ~Prof. Dr. Jan-Henrik Haunert\\
					& ~Prof. Dr. Alexander Wolff
					\vspace{4pt}\\
\multicolumn{2}{l}{Defended on December 21, 2017}					
\end{tabular}
}



\CoverDesign{Max Mustermann}
\FreeTextDesign{a free text for design}

\ISBNprint{978-3-95826-XXX-X}
\ISBNonline{978-3-95826-XXX-X}
\URN{urn:nbn:de:bvb:20-opus-XXXXXX}

\Acknowledgments{%
  \textbf{AW: I would put the acks at the end of the into or just in
    front of the bibliography.}

I would like to thank my supervisor, 
Prof.\ Dr.\ Alexander ``Sascha'' Wolff.
Sascha has been very patient with me.
When I first came to Germany, I could barely speak English.
Sascha had to put a lot of effort into understanding me
when we were doing research.
He always encouraged me to ask questions.
The research work was challenging,
but I enjoyed working in his group.
I was lucky to have him as my supervisor.
He even negotiated with my landlords.

I also thank my second supervisor, Prof.\ Dr.\ Jan-Henrik Haunert.  In
his lecture Algorithms for GIS, he taught my many fundamentals of
GIScience.  Later he moved to other universities and invited me the
visit him there.  He taught me the ``optimization mantra'' (explain?).

I would like to thank Prof.\ Dr.\ Min Deng
for helping me get the chance of
pursuing my PhD in University of W\"urzburg.

AW: Don't forget Guillaume Touya!

I am grateful to Dr.\ Krzysztof Fleszar for 
helping me a lot during my PhD study.
Whenever he found out that I had problems to understand a paper,
he read that paper and then explained it to me.
%On the day of my defense, he sent his last set of comments 
%about my slides at about 3:00 am.
He helped me move twice.
On the day before my defense, he worked as hard as me
to read my slides and to give me feedback.
He has always invited me to join his parties, 
I am grateful to him for his invitations. 

AW: Thomas also did some proofreading for you, didn't he?

I am very grateful to the University of W\"urzburg 
because I had the possibility to freely take many courses.
In order to gain sufficient credits for 
obtaining a PhD degree of Computer Science,
I took some courses
(e.g., \emph{Theoretical Computer Science}
and \emph{Algorithms for Data Structures})
from our own faculty, 
Faculty of Mathematics and Computer Science.
%
In order to improve my language skills,
I took many courses
offered by the Language Center
(e.g., \emph{English for the Natural Sciences} 
and \emph{General Language Exercise of German})
and by Faculty of Arts 
(e.g., \emph{Introduction to English Linguistics}).

It was nice that our group always had lunch together
so that we had plenty of chances to learn from each other.
Indeed, I sometimes took advantage of these lunches
to get suggestions regarding my research work.

thanks to my girlfriend.

%My sincerest thank is meant to my enthusiastic supervisor, 
%[Name]. My PhD has been an amazing experience and I thank 
%[Name] wholeheartedly, not only for his tremendous academic 
%support, but also for giving me so many wonderful 
%opportunities. I am particularly indebted to [Name] for his 
%constant faith in my lab work, all his contributions of 
%time, ideas, and funding to make my Ph.D. experience 
%productive and stimulating. 
%I am also hugely appreciative of [Name], especially for 
%sharing his (computing???) expertise so willingly, and for 
%being so dedicated to his role as my secondary supervisor, 
%and for his support when so generously hosting me in [City].
%A special acknowledgement goes to YOUR Collaborative team 
%[Name]. I praise the enormous amount of help and support by 
%[Name] throughout my stay in Paris. 
%Dozens of people have helped and taught me immensely at 
%Wuerzburg University. Special mention goes to cristoph??, 
%[Name], and [Name], for going far beyond the call of duty. 
%To [Name], for encouraging me to embark on the computer 
%science\Geography?? path, and for inspiring conversations. 
%And to [your other members], for nurturing my enthusiasm for 
%computer science. The group has been a source of friendships 
%as well as good advice and collaboration. I have very fond 
%memories of Squash time here. 
%For this dissertation I would like to thank my reading 
%committee members: [Name], [Name], and [Name] for their 
%time, interest, and helpful comments. 
%My time at Wuerzburg was made enjoyable and enriched in 
%large part due to the many friends and groups that became a 
%part of my life. I am grateful for time spent with flatmates 
%and friends, for my Football Team, Hotpot group, and Skiing 
%buddies. 
%Lastly, I would like to thank my family for all their love 
%and encouragement. For my parents who raised me with a love 
%of science and supported me in all my pursuits. And most of 
%all for my loving, supportive, and encouraging fiancée Fang 
%whose faithful support during the final stage of this Ph.D. 
%is so appreciated. Thank you.
}
\Preface{%

  In his thesis, Dongliang Peng investigates computational aspects of
  a fundamental problem in cartography called \emph{generalization}.
  Cartographic generalization is the problem that has to be solved
  when producing a small-scale map (say, at scale 1:100,000) from data
  collected at a larger scale (say, 1:10,000).  Traditionally, this
  has been a labor-intensive task that required highly-skilled
  cartographers with expertise in both geodesy and map-making.  How
  should a row of ten houses be represented if the distances between
  them are too small to be represented correctly at the desired scale?
  By fewer, say, five houses?  Or rather by a block of houses?  

  Dongliang attacks this and several other special cases of
  cartographic generalization in a modern setting.  Think of online
  maps that users can (and want to!) zoom and pan with their finger
  tips.  Other than traditional paper maps that were produced and
  printed for a fixed, small number of scales, online maps should
  ideally be available at \emph{any} scale (within a reasonable
  interval).  The problem of computing and displaying maps at
  arbitrary scales is called \emph{continuous generalization}.  Some
  aspects of continuous generalization (such as the example with the
  houses above) are inherently discrete, others (such as a wiggly
  river) can be treated more easily in a continuous fashion.
  
  In several chapters of this thesis, Dongliang approaches continuous
  generalization problems in an innovative, yet formal way by using
  powerful tools from mathematical optimization such as integer linear
  programming.  The main advantage of such an approach is that it
  helps to get the \emph{model} right.  Many problems can be cast into
  integer linear programs, which allows us to solve at least small
  examples to optimality~-- according to our model.  If these
  solutions don't turn out as expected, we know that our problem
  formulation, our model is wrong~-- not the algorithm.  In contrast,
  when using only heuristics, solutions can come with errors of two
  types; those caused by the heuristic and those caused by the model.
  This makes it hard to understand and fix them.

  In Chapter~2, Dongliang compares two potentially exact methods for
  the well-known area-aggregation problem.  He asks for an optimal
  sequence of operations that aggregates, step-by-step, many
  ``patches'' of a detailed map into a single region of a coarser map.
  Dongliang uses a graph-based model in which he finds shortest paths
  (aka optimal aggregation sequences) using the A$^\star$-algorithm
  and integer linear programming, which turns out to be much slower.
  He also compares the A^$\star$ algorithm to an obvious greedy
  approach, which is surprisingly good given its simplicity.  He also
  identifies a problem with his model.

  Chapter~3 treats a related problem: how to best draw the boundaries
  of two levels of administrative boundaries when zooming out?  The
  input consists of the two drawings at start and target scale, and
  the task is to mediate between the two continuously in a
  topologically safe way, that is, without introducing
  self-intersections or other degeneracies.  Dongliang finds the first
  topologically safe way of solving this problem; he uses
  \emph{compatible triangulations} (a tool from computational
  geometry) to define a correspondence between the fine and the coarse
  drawing.  Unfortunately, the existing algorithm for compatible
  triangulations sometimes introduces strong distortion locally.  A
  way out may consist in choosing the Steiner points for the
  compatible triangulations more carefully, but this is left as an
  open problem.

  In Chapter~4, Dongliang deals with the generalization of buildings.
  He shows how to generalize building footprints continuously such
  that well-defined blocks of buildings appear when the user of a
  digital map zooms out far enough.  Here, only the building
  footprints are given.  The approach animates a growing-process
  between the current scale and a target scale that can be set by the
  user.  The algorithm computes a drawing for the target scale by a
  wisely chosen sequence of dilations and erosions (which have been
  used for building simplification before).  In order to animate the
  growing-process in a continuous fashion, the buildings are expanded
  in a simple way: by moving their boundaries at constant speed and by
  clipping any part that leaves the target footprint.  In order to
  guarantee a certain minimum distance between two buildings, the
  algorithm builds bridges between close-enough buildings.  The
  computation of the bridges is based on a minimum spanning tree of
  the buildings.  The resulting animations look very natural.

  In Chapter~5, Dongliang explores an important subproblem that often
  appears in continuous generalization: how to ``morph'' a polygonal
  chain from a start to a target scale.  He uses an existing algorithm
  to define a correspondence between the two chains and then computes
  trajectories for each pair of corresponding vertices.  The aim is to
  find trajectories such that the lengths of the segments and the
  angles between them change in a uniform and continuous fashion~-- if
  possible.  Dongliang applies least squares adjustment in order to
  gradually move the chain from its start configuration to the target
  chain.  While the results on real-world data look quite good,
  Dongliang also found artificial examples where self-intersection and
  numerical problems occur.  Hence, least squares is probably not the
  ideal method for this problem.

  Finally, in Chapter~6, Dongliang presents a case study to highlight
  the difficulties when working with geographic data naively.  As a
  concrete example, he considers the problem of finding, in a set of
  $n$ points in the plane, all pairs of points that are closer to each
  other than a given threshold~$\varepsilon$.  He compares three
  approaches for this problem, the text-book approach based on the
  sweep-line paradigm, a Delaunay-triangulation-based approach, and a
  simple grid-based approach.  While the grid-based approach wins in
  terms of runtime, the different implementations of the sweep-line
  algorithms are what makes the comparison interesting~-- a lot
  depends on how the library methods are implemented, so the
  programmer should always read the fine print.

  In his thesis, Dongliang exemplifies the optimization-based approach
  at various continuous generalization problems and demonstrates the
  strength of this approach.  I~see this as a very valuable
  contribution to GIScience where very all too often, problems are not
  modeled properly and then algorithms are devised that happen to
  produce good-looking results on a small set of benchmark instances.
  I~hope that many readers of this thesis will get inspired by
  Dongliang's way of tackling spatial problems!
	
}
\Abstract{	
Maps are the main tool to represent geographical information. 
As geographical information is usually scale-dependent, 
users need to have access to maps at different scales.
In our digital age, people often zoom in or out 
to read maps at different levels.
As discrete changes tend to distract users,
smooth changes are preferred.
This is why some digital maps such as Google Maps 
are trying to make zooming as continuous as they can.
The process of producing maps at different scales
with smooth changes 
is called \emph{continuous map generalization}.
Ideally, there
should be no discrete change in 
continuous map generalization. 
However, the term is also used
when the discrete changes are small enough not to be noticed.

In order to produce maps with high quality,
cartographers often take into account requirements.
These requirements are transferred to models 
in map generalization.
Optimization for map generalization is important 
not only because it finds optimal solutions in the sense of the 
models,
but also because it evaluates the quality of the models.
Optimization becomes more delicate
when we deal with \emph{continuous} map generalization.
In this area, there are requirements 
not only for a specific map, 
but also for relations between maps. 
%
My PhD thesis is about continuous map generalization
based on optimization. 
First, we find optimal sequences for aggregating land-cover 
areas (see Figure 1).
We compare the \Astar algorithm and integer linear programming
in completing this task.
%
Second, we generalize county boundaries to
provincial boundaries.
We morph between the two sets of boundaries, 
where dynamic programming is used to 
compute corresponding points between boundaries.
%
Third, we generalize buildings to built-up areas 
by aggregating and growing (see Figure 2).
In this work, we group buildings based on
minimum spanning tree. 
%
Fourth, we define moving trajectories
for morphing between polylines.
We require that both the lengths of edges and the angles 
between edges should change linearly.
As it is impossible to fulfill all the requirements,
we make a compromise between them 
using least-squares adjustment.
%
Fifth, we discuss the performance 
of some data structures 
(e.g., SortedDictionary and SortedSet) 
for a specific spatial problem.
}


%%% Local Variables:
%%% mode: latex
%%% TeX-master: "../dissertation"
%%% End:
