%%%%%%%%%%%%%%%%%%%%%%%%%%%%%%%%%%%%%%
%%%%%%%%%%%%%%%%%%%%%%%%%%%%%%%%%%%%%%
%FINAL VERSION:
%look for numbered but unlabelled equations
%\usepackage{refcheck}%compile twice in order to see the effects



%change the font sizes
\makeatletter
\renewcommand\normalsize{\@setfontsize\normalsize{10.5}{12.5}} %original: 10, 12
\renewcommand\small{\@setfontsize\small{9.5}{11.5}} %original: 9, 11
\renewcommand\footnotesize{\@setfontsize\footnotesize{8.5}{10}} %original: 8, 9.5
\renewcommand\scriptsize{\@setfontsize\scriptsize{7.5}{8.5}} %original: 7, 8
%As tiny is not used, I use it for the fontsize that I used inside figures.
%I use tiny for \captionsetup.
%I can't find a better way.
\renewcommand\tiny{\@setfontsize\tiny{10}{12}}
\makeatother
\footnotesize
\small
\normalsize
\tiny
%\usepackage{caption} %was defined in wup.cls
\captionsetup{labelfont=bf,
    justification=justified %Distribute text evenly between margins
}
\captionsetup[subfigure]{
	justification=centering,
	font=tiny, %I used 10pt when I drew figures
	labelformat=empty,
}


\usepackage[utf8]{inputenc}
\usepackage[T1]{fontenc}

% running footnotes in whole document 
\usepackage{chngcntr} 
\counterwithout{footnote}{chapter} 
% style of footnotes left-justified in justification 
\usepackage[bottom,hang,marginal]{footmisc} 
\setlength{\footnotemargin}{1.3em}
%extra penalty for breaking after first line of a paragraph: 
\clubpenalty = 10000 
%extra penalty for breaking before last line of a paragraph:
\widowpenalty = 10000
%extra penalty for breaking before last line before a display math:
\displaywidowpenalty = 10000

\graphicspath{{images/}}

%\usepackage[ngerman]{babel}
\usepackage[english]{babel} % to write documents in languages


\usepackage{lipsum}  % Loading lipsum to write blind text



\usepackage{amsfonts}  %for \mathbb{Z}
\usepackage{amsmath}  %align equations; text inside math

\usepackage{amsthm} %for using open problem
\theoremstyle{definition}
\newtheorem{open}{Open Problem}

%\usepackage{array}%for making new column types; included
\usepackage{bm} %make a variable bold in math mode 
\usepackage{booktabs} %for using \toprule, \midrule, and 
%\bottomrule in a table
\usepackage{dcolumn} %for \newcolumntype{d}[1]{D{'}{.}{#1}}
\usepackage{float} %option "H" for placing a figure
\usepackage[bottom]{footmisc}% places footnotes at page bottom
\usepackage{gensymb}  % for degree sign
% when including figure files
\usepackage[defaultlines=4,all]{nowidow}%forbid widow or orphan


\usepackage{hyperref}
\hypersetup{linktocpage=true}
\hypersetup{colorlinks=true}
\hypersetup{hidelinks}
%\hypersetup{pdftitle = {An Optimization-Based Approach for 
%		Continuous Map Generalization}}
\hypersetup{pdfauthor = {Dongliang Peng}}
\hypersetup{pdfsubject = {Dissertation}}
\hypersetup{pdfkeywords = {Land-cover area, Administrative boundary, 
Building, Morphing, Data structure, Zooming}}
\hypersetup{pdfdisplaydoctitle = true}
\newcommand{\dlpLinkColor}{darkgray}
\hypersetup{
	linktoc=all,
	colorlinks,
	citecolor=\dlpLinkColor,
	filecolor=\dlpLinkColor,
	linkcolor=black,
	urlcolor=\dlpLinkColor
}

%prevent links in bibliography overflowing into the margin
%https://tex.stackexchange.com/questions/115690/urls-in-bibliography-latex-not-breaking-line-as-expected
%\hypersetup{breaklinks=true}
\usepackage{url}
\urlstyle{same}
%\Urlmuskip=0mu plus 1mu\relax

\usepackage[charter]{mathdesign} %make the font of math charter
\usepackage{multicol}        % used for the two-column index
%\pgfplotsset{compat=1.3} %make the label tight in axis picture

%\usepackage{sansmathfonts}%make the font in math model sansserif
\usepackage{tabularx} %to define the style of a column in a table


%To encourage LaTeX not to move your figure, 
%you may need to loosen its demands.
%https://texfaq.org/FAQ-floats
\renewcommand{\textfraction}{0.01}
\renewcommand{\topfraction}{0.99}
\renewcommand{\bottomfraction}{0.99}
\renewcommand{\dbltopfraction}{0.99} % fit big float above 
%\renewcommand{\topfraction}{.85}
%\renewcommand{\bottomfraction}{.7}
%\renewcommand{\textfraction}{.15}
\renewcommand{\floatpagefraction}{.66}
%\renewcommand{\dbltopfraction}{.66}
\renewcommand{\dblfloatpagefraction}{.66}
\setcounter{topnumber}{9}
\setcounter{bottomnumber}{9}
\setcounter{totalnumber}{20}
\setcounter{dbltopnumber}{9}


\usepackage[disable]{todonotes} %need package tikz
%\usepackage[textwidth=2cm]{todonotes}
\makeatletter %make package todonotes compatible with wup.cls 
\providecommand\@dotsep{5}
\renewcommand{\listoftodos}[1][\@todonotes@todolistname]{%
	\@starttoc{tdo}{#1}}
\makeatother

\usepackage{xspace} %we use \xspace for \Astar

%if a figure is on an empty last page,
%the figure will be placed at the top instead in the middle of 
%the page
\makeatletter
\setlength{\@fptop}{0pt}
\setlength{\@fpsep}{8pt} %8pt between two figures at last page
\makeatother


%style: Loads bibliography style .bbx and citation style cbx
%bibstyle: Loads bibliography style .bbx
\usepackage[backend=bibtex,
style=alphabetic, %style for my thesis
sorting=nyvt, %sort by name, year, volume, title
%bibstyle=authoryear, %no indices before names; year after names
%bibstyle=numeric, %number the references
hyperref=true,%transform citations and back references into 
%clickable hyperlinks
backref=true, %to print back references in the bibliography
backrefstyle=three, %Compress three or more consecutive pages 
%to a range
%,
%citestyle=authoryear,
maxbibnames=9, %the author names appearing in bibliography
maxcitenames=2, %the author names appearing in text
giveninits=true,
%terseinits=true,%remove periods after initials of first names
%doi=true,
%isbn=false,url=false,
eprint=false,
%dashed=false,
useprefix=false,%sort bibliography not using prefix like "van"
natbib=true]{biblatex}
\addbibresource{general/Reference/BibReference.bib}

%suppress "In:" for journal names
%https://tex.stackexchange.com/questions/10682/suppress-in-biblatex
%\renewbibmacro{in:}{}
\renewbibmacro{in:}{%
	\ifentrytype{article}{}{\printtext{\bibstring{in}\intitlepunct}}}

%add commas after journal names
%https://tex.stackexchange.com/questions/249010/adding-comma-after-journal-name-in-biblatex
\DeclareFieldFormat{journaltitle}{\mkbibemph{#1}\isdot}
\renewbibmacro*{journal+issuetitle}{%
	\usebibmacro{journal}%
	\setunit*{\addcomma\space}%
	\iffieldundef{series}
	{}
	{\newunit
		\printfield{series}%
		\setunit{\addspace}}%
	\usebibmacro{volume+number+eid}%
	\setunit{\addspace}%
	\usebibmacro{issue+date}%
	\setunit{\addcolon\space}%
	\usebibmacro{issue}%
	\newunit}


%if a reference has doi, then we print only doi in bibliography;
%otherwise, we print only url
%we don't use eprint
%we must deactivate eprint by using "eprint=false"
\renewbibmacro*{doi+eprint+url}{% 
	\iftoggle{bbx:url} 
	{\iffieldundef{doi}{\usebibmacro{url+urldate}}{}} 
	{}% 
	%	\newunit\newblock 
	%	\iftoggle{bbx:eprint} 
	%	{\usebibmacro{eprint}} 
	%	{}% 
	\newunit\newblock 
	\iftoggle{bbx:doi} 
	{\printfield{doi}} 
	{}}	



%remove the information we don't want to present
\AtEveryBibitem{% Clean up the bibtex rather than editing it
	\clearlist{address}
	\clearfield{date}
%	\clearfield{eprint}
	\clearfield{isbn}
	\clearfield{issn}
	\clearlist{location}
	\clearfield{month}
	\clearlist{language}
	%	\clearfield{series}
	
	\ifentrytype{book}{}{% Remove publisher and editor except for books
		\clearlist{publisher}
		%		\clearname{editor}
	}
}

%%display pagetotal of proceedings
%\DeclareFieldFormat[inproceedings]{pagetotal}{\mkpagetotal[bookpagination]{#1}}
%\renewbibmacro*{chapter+pages}{%
%	\printfield{note}%
%	\setunit{\bibpagespunct}%
%	\printfield{pages}%
%	\setunit{\bibpagespunct}%
%	\printfield{pagetotal}%
%	\newunit}



%%%%----SET THE STYLE OF BIBLIOGRAPHY----%%%%
%Remove the quotation marks for the titles in the reference
\DeclareFieldFormat[article,inbook,incollection,
inproceedings,patent,thesis,unpublished]{citetitle}{#1}
\DeclareFieldFormat[article,inbook,incollection,
inproceedings,patent,thesis,unpublished]{title}{#1}
\DeclareNameAlias{author}{last-first}%lastname before firstname
%\renewcommand*{\revsdnamepunct}{} %remove commas between last 
%and first names


%mke titles sentence case
%https://tex.stackexchange.com/questions/86043/in-biblatex-make-title-sentence-case-but-not-journal-name/86051
%https://github.com/plk/biblatex/blob/dev/tex/latex/biblatex/biblatex.def
\DeclareFieldFormat[article,inbook,incollection,inproceedings,patent,thesis,unpublished]
{titlecase}{\MakeSentenceCase*{#1}}
%keep the names of journals as they are
\newbibmacro*{journal}{%
	\iffieldundef{journaltitle}
	{}
	{\printtext[journaltitle]{%
			\printfield[myplain]{journaltitle}%
			\setunit{\subtitlepunct}%
			\printfield[myplain]{journalsubtitle}}}}
%keep the names of booktitles as they are
\newbibmacro*{booktitle}{%
	\ifboolexpr{
		test {\iffieldundef{booktitle}}
		and
		test {\iffieldundef{booksubtitle}}
	}
	{}
	{\printtext[booktitle]{%
			\printfield[myplain]{booktitle}%
			\setunit{\subtitlepunct}%
			\printfield[myplain]{booksubtitle}}%
		\newunit}%
	\printfield{booktitleaddon}}
\DeclareFieldFormat{myplain}{#1}


%make the prefix of a name capital, e.g., van Dijk -> Van Dijk
%https://tex.stackexchange.com/questions/226893/biblatex-handling-of-dutch-van-prefix
\renewbibmacro*{begentry}{\midsentence} %lower-case in the bibliography
\makeatletter
\AtBeginDocument{\toggletrue{blx@useprefix}}
\makeatother


%General


\newcommand{\revise}[2]{#2}
\newcommand{\mypar}[1]{\bigskip\noindent\textbf{#1.}~}

%Preface: Adminstrative Boundaries
\newcommand{\ml}{\ensuremath{M_+}\xspace}
\newcommand{\ms}{\ensuremath{M_-}\xspace}
%\newcommand*{\mine}[1]{{\color{red}#1}}
\newcommand*{\mine}[1]{{#1}}

\def\circled#1{%
	~#1%
	\pdfliteral{
		q .5 w
		10 0 0 10 -2.5 3.5 cm .05 w .5 0 m
		.5 .276 .276 .5 0 .5 c -.276 .5 -.5 .276 -.5 0 c
		-.5 -.276 -.276 -.5 0 -.5 c .276 -.5 .5 -.276 .5 0 c h
		S Q
	}~%
}


% math-mode version of "l" column type
\newcolumntype{L}{>{$}l<{$}}
% math-mode version of "r" column type
\newcolumntype{R}{>{$}r<{$}}
% math-mode version of "c" column type
\newcolumntype{C}{>{$}c<{$}}
%align numbers by decimal points in table; 
%need package {dcolumn} and {array}
%see http://www.ctex.org/documents/packages/table/dcolumn.pdf 
\newcolumntype{d}[1]{D{'}{.}{#1}}
\newcolumntype{s}[1]{D{'}{~}{#1}}

%x:l, y:c, z:r, with specified width
%place text in a cell at top (b), middle (m), or bottom (p)
\newcolumntype{x}[1]
{>{\raggedright\arraybackslash\hspace{0pt}}m{#1}}
\newcolumntype{y}[1]
{>{\centering\arraybackslash\hspace{0pt}}m{#1}}
\newcolumntype{z}[1]
{>{\raggedleft\arraybackslash\hspace{0pt}}m{#1}}

%math-mode version of "x,y,z" column type
\newcolumntype{X}[1]
{>{\raggedright\arraybackslash\hspace{0pt}}m{$#1$}}
\newcolumntype{Y}[1]
{>{\centering\arraybackslash\hspace{0pt}}m{$#1$}}
\newcolumntype{Z}[1]
{>{\raggedleft\arraybackslash\hspace{0pt}}m{$#1$}}

\newcommand{\dtrm}[2][]{d_{\mathrm{#2}, 
t\ifthenelse{\equal{#1}{}}{}{_#1}}}

%%Area Aggregation
%%**************************************************************
\newcommand{\Astar}{A$^{\!\star}$\xspace}
\newcommand{\AstarTwo}{A$^{\!\star}_{\,200{,}000}$\xspace}
\newcommand{\AstarFour}{A$^{\!\star}_{\,400{,}000}$\xspace}
\newcommand{\ILPOne}{ILP$_{100\,\mathrm{s}}$\xspace}
\newcommand{\ILPTwo}{ILP$_{200\,\mathrm{s}}$\xspace}

\newcommand{\tstar}{\ensuremath{t\xspace}}
\newcommand{\Pistar}{\ensuremath{\Pi}} % 
%\newcommand{\Comp}[1]{c_#1} % 
\newcommand{\comp}[1]{[#1]} % 
\newcommand{\edgenum}[1]{\|#1\|}
\newcommand{\area}[1]{\overline{#1}}
%{\Pi_{\mathrm{start}}(\tstar)}
\newcommand{\Pstart}{\ensuremath{P_\mathrm{start}}\xspace}
\newcommand{\Pgoal}{\ensuremath{P_\mathrm{goal}}\xspace}
\newcommand{\Pnode}{\ensuremath{P_{t,i}}\xspace}
\newcommand{\Psnode}{\ensuremath{P_{s,i}}\xspace}
\newcommand{\Tgoal}{\ensuremath{T_\mathrm{goal}}\xspace}


\newcommand{\e}[1]{\times 10^{#1}}
\newcommand{\fig}{Figure~}
\newcommand{\eq}{Equation~}
\newcommand{\fo}{Formula~}
\newcommand{\sect}{Section~}
\newcommand{\tab}{Table~}
\newcommand{\chap}{Chapter~}
\newcommand{\figs}{Figures~}
\newcommand{\eqs}{Equations~}
\newcommand{\fos}{Formulas~}
\newcommand{\sects}{Sections~}
\newcommand{\tabs}{Tables~}
\newcommand{\chaps}{Chapters~}
\newcommand{\myquad}[1][1]{\hspace*{#1em}\ignorespaces}
\newcommand{\eqquadVariable}{\myquad[6]}
\newcommand{\eqquadCost}{\myquad[2]}
\newcommand{\eqquadConstraintsX}{\myquad[4]}
\newcommand{\eqquadConstraintsYZ}{\myquad[3]}
\newcommand{\eqquadConstraintsC}{\myquad[3]}
\newcommand{\eqquadConstraints}{\myquad[3]}
\newcommand{\eqquadConstraintW}{\myquad[3]}
\newcommand{\inquad}{\qquad}
\newcommand{\inquadC}{\myquad[1]}

%because of \begin{array}
%\newcommand{\embshift}{\hspace{0.5pt}}%for tsas 
%\newcommand{\embshift}{\hspace{-3pt}}%for my thesis(10pt)
\newcommand{\embshift}{\hspace{-2.7pt}}%for my thesis(10.5pt)

%M or N are default tags. 
%The boxes using the same tag will have the same length. 
%The length is the maximum length of all the equations 
%in the boxes that use the same tag; see also 
%https://tex.stackexchange.com/questions/381502/how-to-align-text-in-multiple-separate-align-environments
\usepackage{eqparbox}
\newcommand{\embl}[2][M]
{\eqmakebox[#1][l]{$#2$}}
\newcommand{\embr}[2][N]
{\eqmakebox[#1][r]{$#2$}}
\newcommand{\embld}[2][M]
{\eqmakebox[#1][l]{$\displaystyle#2$}}
\newcommand{\embrd}[2][N]
{\eqmakebox[#1][r]{$\displaystyle#2$}}

%make package refcheck compatible with package hyperref 
\makeatletter
\AtBeginDocument{%
	\@ifpackageloaded{refcheck}{%
		\@ifundefined{hyperref}{}{%
			\let\T@ref@orig\T@ref
			\def\T@ref#1{\T@ref@orig{#1}\wrtusdrf{#1}}%
			\let\@refstar@orig\@refstar
			\def\@refstar#1{\@refstar@orig{#1}\wrtusdrf{#1}}%
			\DeclareRobustCommand\ref{\@ifstar\@refstar\T@ref}%
		}%
	}{}%
}
\makeatother