%General


\newcommand{\revise}[2]{#2}
\newcommand{\mypar}[1]{\bigskip\noindent\textbf{#1.}~}

%Preface: Adminstrative Boundaries
\newcommand{\ml}{\ensuremath{M_+}\xspace}
\newcommand{\ms}{\ensuremath{M_-}\xspace}
%\newcommand*{\mine}[1]{{\color{red}#1}}
\newcommand*{\mine}[1]{{#1}}

\def\circled#1{%
	~#1%
	\pdfliteral{
		q .5 w
		10 0 0 10 -2.5 3.5 cm .05 w .5 0 m
		.5 .276 .276 .5 0 .5 c -.276 .5 -.5 .276 -.5 0 c
		-.5 -.276 -.276 -.5 0 -.5 c .276 -.5 .5 -.276 .5 0 c h
		S Q
	}~%
}


% math-mode version of "l" column type
\newcolumntype{L}{>{$}l<{$}}
% math-mode version of "r" column type
\newcolumntype{R}{>{$}r<{$}}
% math-mode version of "c" column type
\newcolumntype{C}{>{$}c<{$}}
%align numbers by decimal points in table; 
%need package {dcolumn} and {array}
%see http://www.ctex.org/documents/packages/table/dcolumn.pdf 
\newcolumntype{d}[1]{D{'}{.}{#1}}
\newcolumntype{s}[1]{D{'}{~}{#1}}

%x:l, y:c, z:r, with specified width
%place text in a cell at top (b), middle (m), or bottom (p)
\newcolumntype{x}[1]
{>{\raggedright\arraybackslash\hspace{0pt}}m{#1}}
\newcolumntype{y}[1]
{>{\centering\arraybackslash\hspace{0pt}}m{#1}}
\newcolumntype{z}[1]
{>{\raggedleft\arraybackslash\hspace{0pt}}m{#1}}

%math-mode version of "x,y,z" column type
\newcolumntype{X}[1]
{>{\raggedright\arraybackslash\hspace{0pt}}m{$#1$}}
\newcolumntype{Y}[1]
{>{\centering\arraybackslash\hspace{0pt}}m{$#1$}}
\newcolumntype{Z}[1]
{>{\raggedleft\arraybackslash\hspace{0pt}}m{$#1$}}

\newcommand{\dtrm}[2][]{d_{\mathrm{#2}, 
t\ifthenelse{\equal{#1}{}}{}{_#1}}}

%%Area Aggregation
%%**************************************************************
\newcommand{\Astar}{A$^{\!\star}$\xspace}
\newcommand{\tstar}{\ensuremath{t\xspace}}
\newcommand{\Pistar}{\ensuremath{\Pi}} % 
%\newcommand{\Comp}[1]{c_#1} % 
\newcommand{\comp}[1]{[#1]} % 
\newcommand{\edgenum}[1]{\|#1\|}
\newcommand{\area}[1]{\overline{#1}}
%{\Pi_{\mathrm{start}}(\tstar)}
\newcommand{\Pstart}{\ensuremath{P_\mathrm{start}}\xspace}
\newcommand{\Pgoal}{\ensuremath{P_\mathrm{goal}}\xspace}
\newcommand{\Pnode}{\ensuremath{P_{t,i}}\xspace}
\newcommand{\Psnode}{\ensuremath{P_{s,i}}\xspace}
\newcommand{\Tgoal}{\ensuremath{T_\mathrm{goal}}\xspace}


\newcommand{\e}[1]{\times 10^{#1}}
\newcommand{\fig}{Figure~}
\newcommand{\eq}{Equation~}
\newcommand{\fo}{Formula~}
\newcommand{\sect}{Section~}
\newcommand{\tab}{Table~}
\newcommand{\chap}{Chapter~}
\newcommand{\figs}{Figures~}
\newcommand{\eqs}{Equations~}
\newcommand{\fos}{Formulas~}
\newcommand{\sects}{Sections~}
\newcommand{\tabs}{Tables~}
\newcommand{\chaps}{Chapters~}
\newcommand{\myquad}[1][1]{\hspace*{#1em}\ignorespaces}
\newcommand{\eqquad}{\myquad[6]}
\newcommand{\inquad}{\qquad}

%M or N are default tags. 
%The boxes using the same tag will have the same length. 
%The length is the maximum length of all the equations 
%in the boxes that use the same tag; see also 
%https://tex.stackexchange.com/questions/381502/how-to-align-text-in-multiple-separate-align-environments
\usepackage{eqparbox}
\newcommand{\embl}[2][M]
{\eqmakebox[#1][l]{$#2$}}
\newcommand{\embr}[2][N]
{\eqmakebox[#1][r]{$#2$}}
\newcommand{\embld}[2][M]
{\eqmakebox[#1][l]{$\displaystyle#2$}}
\newcommand{\embrd}[2][N]
{\eqmakebox[#1][r]{$\displaystyle#2$}}