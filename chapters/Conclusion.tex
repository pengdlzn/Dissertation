\chapter{Conclusion}
\label{chap:Conclusion}

In this thesis, we have studied four topics of 
continuous map generalization.
These topics include area aggregation, 
morphing between administrative boundaries,
building generalization, and
defining trajectories for morphing polylines.
The intermediate-scale results of all the four methods
may be used as valid maps.
Also, we have presented some lessons that 
we learned from implementing our algorithms.
In order to achieve continuous map generalization 
of high quality, 
we have integrated optimization into our methods.

A number of methods have been developed for
continuous map generalization.
As maps are complicated,
we need to develop more methods 
in order to fully automate continuous map generalization.
Besides, an urgent issue is to unify the methods.
Currently, each of our methods deals with 
only one type of feature, i.e., 
land-cover areas, buildings, or administrative boundaries.
A map usually contains many kinds of features.
It would be interesting to work on a complete map,
where we would need to care about the relations 
between different kinds of features.
For example, depending on the situations, 
we may use streets to confine the growing of 
buildings to built-up areas, 
or we may cover the streets with built-up areas 
because we want to aggregate several blocks into one.

The main goal of continuous map generalization is to provide 
users with better zoom experience.
Hence, more usability tests are needed to see 
if the results of continuous map generalization 
are indeed better than other zoom strategies.
We planed to compare our work about building generalization 
(see \chap\ref{chap:Bldg}) 
with the results of \emph{progressive block graying} 
\parencite[see][]{Touya2017Progressive}.
As they and we have obtained results 
based on different strategies, 
it is possible (and necessary) to make a comparison 
based on some usability test.
To this end, we should design some tasks 
(e.g., pointing out the 
position of a building during zooming out), 
and then ask participants to complete these tasks
\parencite[e.g.,][]{Midtbo2007}.
Then we can compare the time and the accuracies
of users' working on the tasks.
Moreover, \textcite{Suba2016Usability} 
have already made a plan for 
testing vario-scale maps, which is related to our research.


It is important to generate intermediate-scale results 
that have simple relations among each other.
These simple relations often result in 
small extra storage and real-time visualization.
Our intermediate-scale maps of area aggregation 
(see \chap\ref{chap:AreaAgg}) 
have simple relations.
These maps can be stored easily
using the generalized area partitioning tree (GAP-tree).
In order to display our morphing between 
two sets of administrative boundaries, 
we only need to store the two datasets 
and a set of relations between the two datasets. 
When users are zooming,
we only need to interpolate linearly between 
the two sets of administrative boundaries 
on the fly\footnote{For some examples, see 
\url{www1.pub.informatik.uni-wuerzburg.de/pub/data/agile2016/}.}.
Our intermediate-scale maps of buildings 
(see \chap\ref{chap:Bldg}), 
however, have no simple relations.
Currently, we clumsily store many intermediate-scale maps
and then transfer all these maps to users when they are zooming.
This strategy may cause time lags.
We should either improve our method of generating 
intermediate-scale maps 
or find a way to construct simple relations 
between our current intermediate-scale maps.

We want to make our prototype more easily accessible 
for other scientists.
Although our prototype is open access on GitHub\footnote{See 
\url{https://github.com/IGNF/ContinuousGeneralisation}.},
one needs to install and configure many libraries in order to 
run the prototype on their own computers.
We wish to put our prototype on a server and allow 
other scientists to access our prototype through a website
software as a service.
The computation should be on the server,
while the users can input their own data and get the output.
Also, We would like to make our prototype more user-friendly.
In the following, we show concrete open problems.

In \chap\ref{chap:AreaAgg}, 
we have shown how to compute optimal
aggregation sequences for % maps with land-cover areas.
land-cover maps.
We tried solving this problem by a greedy algorithm,
the \Astar algorithm,
and a method based on integer linear programming.
% I'd use "ILP" only for "integer linear program", 
% not for the "integer linear programming" technique.
For the \Astar algorithm, 
we have a good estimation for the cost of type change, 
which helps a lot to reduce the search space, 
but our estimation for the cost of shape 
(compactness or edge length) is rather poor.
The ILP-based algorithm, on the other hand, 
could not even find \emph{feasible} solutions 
for some of our test instances.
We have the following open problems.
 
\begin{open}
How to find a better estimation for the cost of shape 
(compactness or edge length)?
\end{open} 

\begin{open}
Do we need to consider other aspects in our cost function?
\end{open}

\begin{open}
Is there an ILP formulation that can be solved faster than ours?
\end{open}

In \chap\ref{chap:Admin}, 
we have continuously generalized 
county boundaries to provincial boundaries by morphing.
Recall that there are more boundaries on the county map.
In order to morph a county boundary 
that is not at the same time a provincial boundary,
we generate the corresponding boundary based on
compatible triangulations.
We observed that some of the boundaries 
that we generated using are heavily distorted.  
These distortions lead us to the following open problems.

\begin{open}
  How much can we reduce the distortion 
  if we use common chords as many as possible
  when constructing compatible triangulations?
\end{open} 

\begin{open}
  Can we efficiently minimize the distortion 
  over all generated boundaries?
\end{open} 

\begin{open}
  How to properly evaluate our continuous generalization 
  of administrative boundaries?
\end{open}

In \chap\ref{chap:Bldg}, we continuously 
generalized buildings to built-up areas
by aggregating and growing.
We managed to produce a sequence of maps 
in which we grew and simplified the buildings.
We compared the numbers of buildings 
with the numbers implied 
by the law of \textcite{Topfer1966}.
Although this law has been criticized by 
\textcite{Jiang2015Fractal},
who proposed calculating the numbers of objects
according to the fractal nature of maps,
there is no generally accepted formula for 
computing the numbers so far.  
Therefore, the following questions remain open.

\begin{open}
  For a given map and a scale, 
  how many buildings should be kept after generalization?
\end{open}

\begin{open}
  Again, for a given map and a scale, 
  how much total area of buildings should be kept 
  after generalization? 
  What about the total number of edges?
\end{open}

\begin{open}
  How to avoid the lengthy buildings generated by our method?
\end{open}

\begin{open}
  How to design a meaningful user study 
  to evaluate our and other approaches 
  to continuous building generalization?
\end{open}

In \chap\ref{chap:Morph}, 
we have introduced a morphing method for polylines 
that tries to change angles and edge lengths linearly over time. 
Our approach is based on least-squares adjustment;
it allows us to handle both hard and soft constraints. 
Our first results are promising. 
Still, there are open problems. 

\begin{open}
  How can we ensure that our LSA-based method always 
  converges to a good solution?
\end{open}

\begin{open}
  Do we have better models for the changes of 
  angles and edge lengths, instead of using linear functions?
\end{open}

\begin{open}
  How to avoid self-intersections of a polyline during a morph?
  How to avoid a polyline intersecting other polylines?
\end{open}

In \chap\ref{chap:DataStr}, we have compared three methods
for finding close-point pairs, i.e., a sweep-line algorithm, 
an algorithm based on the Delaunay triangulation, 
and a grid-based algorithm.
The grid-based algorithm was 
the clear winner of our comparison.
However, the sweep-line paradigm 
can be used to solve many problems,
e.g., computing the Voronoi diagram 
\parencite{Fortune1987Voronoi}.
The following questions remain open.

\begin{open}
  What is the smallest radius such that 
  the algorithm based on Delaunay triangulation
  finds \emph{all} pairs of close points?
\end{open}

\begin{open}
  Is the grid-based algorithm faster than other algorithms, 
  including those not mentioned in the chapter, 
  in finding close edges or close polygons?
\end{open}

\begin{open}
  Will the grid-based algorithm be the winner
  when we extend the three methods to higher dimensions?
\end{open}